\documentclass[twocolumn, prl]{revtex4-1}
\usepackage[pdftex]{graphicx}
\usepackage{subfig}
\usepackage{amsmath}

\begin{document}

\newcommand{\be}{\begin{equation}}
\newcommand{\ee}{\end{equation}}


\title{Symmetry constrained machine learning}
\date{\today}

\author{Doron L. Bergman}
\affiliation{MEGA LLC}
\email[E-mail me at: ]{doronator@gmail.com}

\begin{abstract}
Symmetry, a central concept in understanding the laws of nature, has been used for centuries in physics, mathematics, and chemistry, to help make mathematical models tractable. Yet, despite its power, symmetry has not been used extensively in machine learning, until rather recently. In this article we show a general way to incorporate symmetries into machine learning models. We demonstrate this with a detailed analysis on a rather simple real world machine learning system - a neural network for classifying handwritten digits, lacking bias terms for every neuron. We demonstrate that ignoring symmetries can have dire over-fitting consequences, and that incorporating symmetry into the model reduces over-fitting, while at the same time reducing complexity, ultimately requiring less training data, and taking less time and resources to train.
\end{abstract}

\maketitle


\section{Introduction}
\label{Sec:Intro}

% symmetry is awesome in physics!

Symmetry, together with understanding the relevant microscopic degrees of freedom and interactions, are the core ingredients behind our understanding of many body systems in the physical world.

% symmetry has not been used in ML much

In statistics and computer science, traditionally there has been little use of symmetry. The recent emergence of convolutional neural networks applied to computer vision\cite{lecun1999object, lecun1990handwritten, lecun1998gradient, krizhevsky2012imagenet, lee2009convolutional, lecun2015deep}, where translational symmetry of images is used to constrain the form of neural network models, is a first demonstration of the vast potential in building symmetries into machine learning models.

% symmetry can help in ML too!

One of the prime challenges for any sufficiently rich model is over-fitting. The physics approach to over-fitting has been to build a more constrained theory, with as few free parameters as possible. The statistics community has not had that luxury, and so they have attacked the over-fitting problem with various methods. Symmetry provides a fruitful avenue to constrain rich models, eliminating superfluous free parameters. By lowering the number of free parameters, models are not only less prone to over-fitting, they also require less data to train, and can be trained and run predictions faster.

% can we find enough symmetry in ML?

There are many obvious symmetries in vision problems - translation, rotations, and mirror images - essentially most portions of affine transformations\cite{gens2014deep, dieleman2016exploiting, cohen2016group, henriques2016warped}. Even some forms of approximate symmetry exist\cite{kiddonsymmetry} - the fact that sentences in some languages are still somewhat understandable even if the words are re-arranged, suggests natural languages have an approximate word permutation invariance. This may explain why bag of word methods are often successful, despite not retaining the full information about the word order. 
%It is the author's belief that many symmetries may be hiding in various machine learning applications, they simply need to be thought about carefully.

% Symmetry for all ... ML models
Given the benefits of combining symmetry into machine learning models, it would be useful to have a simple way to incorporate symmetries into any machine learning model. In this paper we address this problem. In short, if inputs are transformed into features that are invariant under the full symmetry group, then the model outputs will automatically be invariant as well. Are we missing any information in this way? No. Any feature that is not invariant under symmetries of the system, will spoil the model invariance, at least for some set of weights. A successful symmetry invariant feature map will be a truncated set of features that suffices to capture enough of the information to make the model successful.

% paper outline
In order to demonstrate the importance and power of symmetry invariant feature maps, we consider in this paper a very simple example - recognizing hand written digits using neural networks with bias terms absent. In section~\ref{Sec:theory}, we describe the toy model we work with to demonstrate our ideas. In section~\ref{Sec:empirics} we describe results of our empirical computer experiments confirming the mathematical theory, and demonstrating the power and ease of implementation of symmetry invariant feature maps. Finally we close with some closing remarks in section~\ref{Sec:conclusions}.


%Anecdotal evidence from the web that people use image mirroring to trick computer vision systems from identifying the content of the image:

% https://www.quora.com/Do-videos-with-reversed-image-frames-avoid-watermark-detection-by-YouTubes-copyright-identification-system

% https://www.reddit.com/r/todayilearned/comments/hqq6k/til_you_can_flipmirror_a_youtube_video_to/

\section{Theory}
\label{Sec:theory}
\subsection{Initial NN model}
\label{Sec:dumb_NN}

For the sake of simplicity of the analysis, we will assume that our images are comprised of pixels with grayscale values scaled to the 
range $[-1,+1]$. We shall also use a hyperbolic tangent function as the nonlinear neuron model, rather than a logistic function, again to map the internal features of a neural network (NN) to the range of values $[-1, +1]$. The neurons in this case are odd functions $\tanh(-x) = - \tanh(x)$.
Finally, we shall not use any biases in any neuron in the system. Denoting the outputs from layer $n$ by $z^{(n)}_i$, the NN is built out of neurons performing the following function
\be
z^{(n+1)}_i = g\left(\sum_j w^n_{i, j} z^{(n)}_j\right)
\; ,
\ee
where $g(z)$ in our case is $\tanh$.
The first input layer takes in the image pixels $z^0_i = x_i$. For classifying the digits, a final softmax layer produces 10 outputs which are the normalized probabilities for the image to be any one of the 10 digits
\be
p_{\alpha} = P(y=\alpha | x) \sim \exp\left[ \sum_i u_{\alpha, i} z^{(N-1)}_i \right] = {\tilde p}_{\alpha}
\ee
with the normalization
\be
p_{\alpha} = \frac{{\tilde p}_{\alpha}}{\sum_{\alpha} {\tilde p}_{\alpha} }
\; .
\ee

\subsection{Issues with symmetry}

{\bf TODO: demonstrate that NN without biases can get very high accuracy on just the original dataset. Do so both with 2-layer RBM+logistic system, and on more general NN with multiple layers}

As demonstrated here (link...), the NN model above, with just $N=2$ layers, can achieve rather high classification accuracy of the 10 digits.
It is trained on an image dataset of black digits on a white background, with the first layer trained as a restricted Boltzmann machine, and the softmax layer trained as a simple logistic classifier.

However, let us challenge the system. Consider the images with the black/white colors inverted, such that the digits are now white on 
a black background. The geometric contrast is exactly the same, and so we would hope that a good machine learning system captures the geometric features of the digit themselves, and would be able to handle the inverted colors case well.

One can test this empirically (add link to gist online demonstrating this in python), and find that the classification accuracy is quite poor, in some cases going below the accuracy of random selection (!). We explore the source of this poor performance, and prove that it is a general effect, rather than a mere anecdote.

Under the inversion symmetry, $x_i \rightarrow - x_i$, every layer except the final softmax layer is odd in its 
input. Therefore under this symmetry we get all the neuron values flipping their sign $z^{(n)}_i \rightarrow - z^{(n)}_i$. Finally, the unnormalized probabilities transform as
\be
{\tilde p}_{\alpha} \rightarrow \exp\left[ - \sum_i u_{\alpha, i} z^{(N-1)}_i \right] = 1/{\tilde p}_{\alpha}
\; .
\ee
The un-normalized probabilities are inverted. For any given image in the original dataset, the digit the model thinks has the highest probability to match the image, is also the digit with the lowest probability to match the inverted image. For every image the model identifies correctly, the inverted image
gets identified incorrectly.

Let us divide the original sample set into two subsets - one for which the model correctly identifies the digit $A$, and $B$ for the remainig samples for which the model identifies the wrong digit. The accuracy rate of the model over this dataset is 
\be
R = \frac{|A|}{|A| + |B|}
\; .
\ee
All the images in $A$, when inverted, will yield the wrong digit from the model. Denoting the subset of the images for which the model identifies the digit correctly on the inverted image as ${\bar A}$, and the remaining as ${\bar B}$, we note that $A \subseteq {\bar B}$. The accuracy of the model on the inverted images is then
\be
{\bar R} = \frac{|{\bar A}|}{|{\bar A}| + |{\bar B}|} = 1 - \frac{|{\bar B}|}{|{\bar A}| + |{\bar B}|} \leq 1 - R
\; .
\ee
Written more symmetrically as $R + {\bar R} \leq 1$ this demonstrates that the accuracy over the original dataset directly constrains the accuracy that can be had on the inverted images. If the original model is accurate more than half the time over this dataset, $R>0.5$, then ${\bar R} \leq 1 - R < 0.5$. Higher accuracy on the original dataset comes directly at the expense of accuracy on the inverted images. 

This is not a very happy state of affairs. 

The weakness of this model stems from two sources - the lack of biases for each neuron, and the features we choose to work with. If we were to re-introduce the biases, what we have proved above no longer holds strictly, but since the accuracy is a smooth function of the biases, if all the biases are infinitessimaly small, the upper limit on the accuracy can be only infinitessimally larger than in the vanishing biases case.

The standard approach in the machine learning community at present would be to train the model on both the original dataset, and the inverted copy of the dataset. If care is taken to avoid overfitting, and learn the right features of the dataset, then the model should enjoy high accuracy. From our empirical attempt we find that doing so may be challenging.
{\bf TODO: show poor performance when training on both halfs of the dataset}
Our conclusion is that the system is overfitting, and rather than learning the important geometric features defining the digits, it is learning features peculiar to their images in black on a white background.

Let us consider an alternative approach, where rather than re-introducing the biases, we change the features we feed to the model. In particular, let us choose features that are invariant under the inversion symmetry.

{\bf TODO: needs revision from here}

A model such as the one we are considering here would usually be trained on the original dataset, with only black digits on a white background. Once confronted with the reality of poor performance on the inverted images, a common approach is to simply add the additional manipulated samples to the training dataset.

However given the demonstrated $R + {\bar R} \leq 1$ constraint, if the training fully converges onto the best model, then that would suggest 
the constraint is saturated at $R + {\bar R} = 1$, and if the images and the inverted images are given equal weight in determining the model, one would expect $R = {\bar R} = 0.5$, a rather limited ceiling on the potential accuracy of the model.

\subsubsection{Does the bias matter?}

Will re-introducing the bias terms save the NN model?

\subsection{A model incorporating symmetry}
\label{Sec:smart_rbm}

What if we could construct the system to begin with to have the inversion symmetry built in to the model?
The simplest way to do this is to map the inputs (the pixels of the image) $x_i$ onto features that are invariant under the inversion symmetry.
Let us consider a generic smooth feature mapping. It can be expanded in a Taylor series
\be
f_a({\bf x}) = \sum_{n_1, n_2, \ldots} A^{(a)}_{n_1, n_2, \ldots} x_1^{n_1} x_2^{n_2} \ldots
\; .
\ee

As a significant simplification, let us assume that all the inputs have the extreme values $\pm 1$ - just black or white, with no shades of gray in the image. The inputs then satisfy the algebra $x_i^2 = 1$, and the general Taylor expansion simplifies to the so-called 
interaction terms
\be
f_a({\bf x}) = A^{(0)} + \sum_i A^{(1)}_i x_i + \sum_{i, j} A^{(2)}_{i, j} x_i x_j + \sum_{i, j, k} A^{(3)}_{i, j, k} x_i x_j x_k + \ldots
\; .
\ee
The feature maps that are invariant under the inversion symmetry are those functions even in the inputs $x_i$ - only the even power terms are allowed.
\be
f_a({\bf x}) = A^{(0)} +  \sum_{i \neq j} A^{(2)}_{i, j} x_i x_j + \sum_{i, j, k, \ell} A^{(4)}_{i, j, k, \ell} x_i x_j x_k x_\ell + \ldots
\; ,
\ee
where the indices in each sum are all unique. 
The lowest order feature maps that are non-trivial are the quadratic maps
\be
f_a({\bf x}) = A^{(0)} +  \sum_{i \neq j} A^{(2)}_{i, j} x_i x_j
\; .
\ee

If instead of the $x_i$ inputs, we fed the inversion symmetry invariant features $\chi_{i,j} = x_i x_j$ into the NN model, we would already know that the classification is perfectly invariant under the inversion symmetry.

However, given $x_{1 \ldots M}$ pixels, the number of $\chi_{i,j}$ features is $M(M-1)/2$, a far larger number of features to the NN model. The comparison is not fair - there would be many more free parameters using these features than in the original model. Generally a model with more free parameters can better fit a given dataset, albeit with an increased danger of overfitting.
To level the playing field we will choose a subset of the possible features with a similar number of free parameters
\be
\chi_i =  x_i x_{i + {\hat e}_1}
\; ,
\ee
where $i + {\hat e}_1$ denotes the pixel immediately to the right of the pixel $i$, and the sum over $i$ is implicitly restricted to those pixels that are not on the right edge of the image. The products $x_i x_{i + {\hat e}_1}$ contain information similar to a gradient of the image pixels since it is +1 when the pixels are of the same color, and -1 if they are different. If we were to look at an image made up of the $\chi_i$ features, it would show the boundaries between the black and white regions of the original image. Restricting just to this subset of features, we start out providing fewer features to the model, since there are only $M - \sqrt{M}$  features $\chi_i$ (assuming square images).

We proceed to test the two different models.

\subsubsection{What if we re-introduce the biases?}

\subsubsection{are these features handpicked to work nicely?}

They are somewhat translationally invariant!

\section{Numerical experiment}
\label{Sec:empirics}


\subsection{Interim conclusion}

What we learn from the analysis above is that we can always engineer features that are symmetry invariant, and simply feed those into an otherwise oblivious machine learning model.



The first symmetry that has been widely incorporated into machine learning models is a limited degree of translational symmetry, in the form of convolutional neural networks, and there are some very recent attempts at incorporating a limited form of rotational symmetry into computer vision systems as well (cite...). These symmetries have to be incorporated in a complex way into the machine learning models, since the symmetries aren't perfect. An image has a boundary, which ruins the perfect translational invariance. Using pixels makes continuous rotations of the image difficult to express. Finally, some features of the images are explicitly not invariant under rotations - an image of the digit 6 under a 180 degrees rotation, becomes the digit 9.

If the image were a continuous field, rather than a matrix of pixels, that would make expressing rotations much simpler. If the image had no boundary, but rather had periodic boundary conditions in both the horizontal and vertical direction, it would live on a 2-torus, and have perfect translational invariance.

In theoretical physics, perfect symmetries are used all the time, despite the fact that there too symmetries are rarely perfect. In the case of calculating the electronic band structure in a crystalline solid, one almost always assumes periodic boundary conditions to perform the calculations. The justification is that for a macroscopic solid, the boundary involves a rather minute fraction of the atoms, and therefore has a negligible effect. 
The 2 body gravitational problem used to understand the orbit of the Earth around the sun, is assumed to have perfect spherical symmetry, and then it can be solved in closed form. This however neglects the presence of the moon, and other planets, which disrupts the spherical symmetry. However there are methods to deal with this difficulty as well. The standard Physics approach is to assume as much symmetry as you can to simplify the problem, and then add the symmetry breaking elements as perturbations to the perfectly symmetric solvable problem. 

Inspired by the approach to imperfect symmetry in Physics, we shall advocate for a similar approach in machine learning models. Instead of the complex approach of ConvNets to imperfect translational invariance of images, we shall assume perfect translational invariance in a "zero order model", and add a relatively small number of features breaking the symmetry, as perturbations to the model.

As a concrete example, consider a grayscale image parameterized by a scalar field $\phi({\bf x})$. Let us construct features with perfect translational invariance

... zero momentum terms...
\be
F_0 = A^{(0)} + A^{(1)} \sum_{\bf x} \phi({\bf x}) + \sum_{\bf a} A^{(2)}_{\bf a} \sum_{\bf x} \phi({\bf x}) \phi({\bf x} + {\bf a})
 + \sum_{\bf a, b} A^{(3)}_{\bf a, b} \sum_{\bf x} \phi({\bf x}) \phi({\bf x} + {\bf a})  \phi({\bf x} + {\bf b}) + \ldots
\ee

Perturbation
\be
F_1 = B \sum_{\text{boundary}} \phi({\bf x})
\ee
The idea here is that if the boundary is all white, then you can translate the image a bit, and the boundary sum will remain the same. It will only change when the digit starts overlapping with the boundary. Those are the only cases we want to avoid predicting - when the image crosses the boundary.

Let us also incorporate the 180 degree rotation that maps 6 to 9, in addition to the translation symmetry. 
The precise origin of the coordinate system for the pixels does not matter, so we pick it to be at the center of the image.
Then, the rotation about the center of the image transforms the grayscale image as 
\be
\phi({\bf x}) \rightarrow \phi(-{\bf x})
\; .
\ee
One can see immediately that an additional constraint must be satisfied by the $F_0$ features for invariance under the 180 rotation.
\be
A^{(2)}_{\bf a} = A^{(2)}_{- {\bf a}}
\; .
\ee
This can be easily expressed by changing $F_0$ a bit
\be
F_0 = A^{(0)} + A^{(1)} \sum_{\bf x} \phi({\bf x}) + \sum_{\bf a} A^{(2)}_{\bf a} \sum_{\bf x} \phi({\bf x}) 
\left[ \phi({\bf x} + {\bf a}) + \phi({\bf x} - {\bf a}) \right]
\ldots
\ee

A model using these features should be trained against labels where 6 and 9 are treated as the same. An additional, much smaller 
machine learning system can then be employed, with additional features that explicitly break the rotation symmetry, to distinguish 
between 6 and 9.


An image has a boundary, which ruins the perfect translational invariance a scalar field has on the 2-torus.


 images comprised of pixels have a grid structure



\section{Conclusions}
\label{Sec:conclusions}

Later

\section{Acknowledgements}

The author would like to thank Miles Stoudenmire, Daniel Malinow, ... 
for feedback on the ideas presented in this manuscript.


\vskip -0.2in

%\bibliography{A-D,E-H,I-L,M-O,P-S,T-Z}
\bibliographystyle{plainnat}
\bibliography{NN_bib}
%\bibliographystyle{TUPREP}



\vskip 0.2in

\end{document}





\section{ideas}
\label{Sec:Ideas}


cite Pedro Domingos

IDEA!

6 and 9 are 180 degrees rotation of each other. Using full rotational symmetry would erase the distinction between them. But a symmetry breaking term would easily differentiate between them - even just one feature.

Another IDEA!

ConvNets have a clumsy form of translational invariance - ther eis a boundary. So the actual translational invariance we take is smaller than the full translational invariance on a 2-torus, and more complicated to implement and think about. So, we propose to use the full 2-torus translation inavariance first, and then apply a small symmetry breaking which will constitute the boundary of the images. Does this even matter?



